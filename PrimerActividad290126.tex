\documentclass[12pt,a4paper]{article}

% --- Paquetes ---
\usepackage[spanish, es-tabla]{babel}
\usepackage[T1]{fontenc}
\usepackage{amsmath, amssymb}
\usepackage{pgfplots}
\pgfplotsset{compat=1.18}
\usepackage{geometry}
\geometry{margin=2.5cm}

% --- Configuración de Formato ---
\usepackage{setspace}
\onehalfspacing % Interlineado de 1.5

\usepackage{titlesec}
% Títulos de secciones a 16 puntos
\titleformat{\section}{\normalfont\fontsize{16}{19}\bfseries}{\thesection}{1em}{}
% Subsecciones a 14 puntos (opcional para jerarquía)
\titleformat{\subsection}{\normalfont\fontsize{14}{17}\bfseries}{\thesubsection}{1em}{}

\begin{document}

% --- Hoja de Presentación ---
\begin{titlepage}
    \centering
    {\bfseries\fontsize{20}{24}\selectfont NOMBRE DE LA ESCUELA \par}
    \vspace{2cm}
    {\bfseries\fontsize{16}{19}\selectfont Actividad: Convalidación \par}
    \vspace{2.5cm}
    {\large \textbf{Nombre del Alumno:} \\ Tu Nombre Completo \par}
    \vspace{0.5cm}
    {\large \textbf{Matrícula:} \\ 12345678 \par}
    \vspace{0.5cm}
    {\large \textbf{Cuatrimestre y Grupo:} \\ 4° - Grupo "A" \par}
    \vspace{2.5cm}
    {\large \textbf{Docente:} \\ Nombre del Profesor \par}
    \vfill
    {\large Jueves, 29 de enero de 2026 \par}
\end{titlepage}

% --- Índice ---
\newpage
\tableofcontents
\newpage

% --- Contenido ---
\section{Introducción}
Este proyecto analiza el comportamiento del consumo de memoria RAM en un servidor que aloja una página web, considerando una carga constante de 50 usuarios y fluctuaciones debidas a horas pico.

\section{Modelado de la Ecuación}
Para modelar el consumo de RAM ($R$) en Gigabytes (GB) respecto al tiempo ($t$) en horas, definimos la siguiente función:
\begin{itemize}
    \item \textbf{Carga base:} $2$ GB (Sistema operativo y servicios base).
    \item \textbf{Consumo por usuario:} $50 \text{ usuarios} \times 0.1 \text{ GB/u} = 5$ GB.
    \item \textbf{Variación por hora pico:} Una función seno para simular el ciclo de carga.
\end{itemize}

La ecuación resultante es:
\[ R(t) = 7 + 3\sin\left(\frac{\pi t}{12}\right) \]
Donde $t \in [0, 24]$ representa las horas del día.

\section{Visualización del Consumo}
\begin{center}
\begin{tikzpicture}
\begin{axis}[
    title={Consumo de RAM en 24 Horas},
    xlabel={Hora ($t$)},
    ylabel={RAM (GB)},
    xmin=0, xmax=24,
    ymin=0, ymax=12,
    xtick={0,4,8,12,16,20,24},
    grid=both,
    width=11cm,
    height=7cm
]
\addplot [domain=0:24, samples=100, color=red, thick] {7 + 3*sin(deg(pi*x/12))};
\end{axis}
\end{tikzpicture}
\end{center}

\section{Resolución de la Integral}
Calcularemos el \textbf{consumo total acumulado} durante las primeras 12 horas mediante la integral definida:

\[ \int_{0}^{12} \left[ 7 + 3\sin\left(\frac{\pi t}{12}\right) \right] dt \]

\subsection{Procedimiento matemático}
1. Separamos la integral:
\[ \int_{0}^{12} 7 \, dt + \int_{0}^{12} 3\sin\left(\frac{\pi t}{12}\right) dt \]

2. Integramos:
\[ \left[ 7t \right]_{0}^{12} + 3 \left[ -\frac{12}{\pi} \cos\left(\frac{\pi t}{12}\right) \right]_{0}^{12} \]

3. Evaluamos en los límites:
\[ (84 - 0) - \frac{36}{\pi} (\cos(\pi) - \cos(0)) \]
\[ 84 - \frac{36}{\pi} (-1 - 1) = 84 + \frac{72}{\pi} \]

4. Resultado final:
\[ \text{Consumo acumulado} \approx 106.92 \text{ GB}\cdot\text{h} \]

\section{Conclusión}
El servidor requiere una gestión de memoria que soporte picos de hasta 10 GB de RAM para garantizar la estabilidad de los 50 usuarios concurrentes en las horas de mayor demanda.

\end{document}